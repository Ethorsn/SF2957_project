\section{Implementation} \label{sec:implementation}
In implementing a Q-learning algorithm for Blackjack, we face a number of choices.
First, as discussed in Sections \ref{sec:stationaryMDP} and \ref{sec:nonstationaryMDP}, is the choice of state space.
Second, the parameters $\alpha$ and $\epsilon$ of the algorithm, which may be functions of the current state $s$ and
the action $a$. Third, the implementation itself---how do we implement the algorithm in code? The first and third points
are interconnected here, so we deal with the second point first, then the other two.

\subsection{Parameter choices for Blackjack Q-learning} \label{sec:paramchoices}
The (possibly) state- and action-dependent learning rate $\alpha_t(s, a) \in (0, 1]$ determines to what extent we update 
the $Q$ function each time we visit state $s$ and take action $a$. A small $\alpha_t$ means that $Q$ changes slowly
from its initial value (which is an additional parameter to choose), and a larger $\alpha_t$ means that we put more
emphasis on the immediate (well, at $t+1$) reward and currently optimal action of the state $s'$ we transition to.
%
Intuitively, our estimate of $Q(s,a)$ should improve with the number of times we have visited $s$ and taken action $a$,
meaning that it becomes less important to update the function in this state-action pair. Vice versa, in state-action pairs
we have not encountered many times before, our estimate of the function value is uncertain, and we should put more emphasis
on the reward we received for taking action $a$, as well as the currently known best action out of the state $s'$ we transition to.
For our Blackjack setup, we may note again that the immediate rewards are 0 in all states except for a terminal state, where the payout is made. 
By convention the function $Q$ is defined to be zero for terminal states. Thus, the update rule \eqref{eq:Qupdate} becomes
% \footnotesize
\begin{align} \label{eq:bjQupdate}
          Q(s_t,a_t)  \leftarrow 
          \begin{cases}
            (1-\alpha_t(s_t,a_t))Q(s_t,a_t) + \alpha_t(s_t,a_t) R_{t+1}, &s_{t+1} \text{ terminal} \\
            (1-\alpha_t(s_t,a_t))Q(s_t,a_t) + \alpha_t(s_t,a_t) \, \underset{a \in A}{\max} Q(s_{t+1},a), &s_{t+1} \text{ non-terminal}. \\
          \end{cases}
\end{align}
% \normalsize 

A choice of $\alpha_t(s_t,a_t)$ that has the intuitive properties mentioned above is $\alpha_t(s_t,a_t) = \#[(s=s_t,a=a_t)]^{-1}$,
i.e.\ the reciprocal of the number of times action $a = a_t$ has been taken in state $s=s_t$. This choice of $\alpha_t$ satisfies
the convergence criteria $\sum_t \alpha_t(s,a) = \infty$ and $\sum_t \alpha^2_t(s,a) < \infty$ for each pair $(s, a)$ outlined in e.g.\ 
\citet{Watkins1992}. Further investigation shows that one can do better than this when the function $Q$ is updated 
asynchronously, i.e.\ one $(s,a)$-pair at a time. \citet{EvenDar2003} show that in this case, the optimal choice is 
$\alpha_t(s_t,a_t) = \#[(s=s_t,a=a_t)]^{-\omega}$, where $\omega \approx 0.77$. This is the choice of $\alpha_t$ we choose,
although it should be noted that the scenario considered by \citet{EvenDar2003} included a discount factor $\gamma < 1$.

Another important choice in implementing Q-learning is the choice of policy---what action should we take in a given state? 
Under certain conditions, our learned $Q$ will eventually guide us to the optimal policy, but this is not the case initially.
If we used the greedy policy of always choosing the action $a$ with the highest value of $Q$ when in state $s$, an unlucky
start may lead us to never explore some states and find (approximately) the true value $Q^*(s,a)$ for all actions $a$ 
available when in state $s$---some perhaps more valuable than the greedy action chosen. Indeed, the second condition for 
convergence to $Q^*$ is that all pairs $(s, a)$ are visited infinitely often, asymptotically \citep{RLDP}.
Commonly, this is done by choosing a random action with some small probability $\epsilon$, rather than choosing the greedy action.
Again, it is reasonable that if we have visited a given state $s$ many times, and also taken all actions our of this state
many times, our estimate $Q(s,a)$ for all actions $a$ available in $s$ should be fairly certain. Thus, there is less need to
choose a random action; we may go with the greedy choice with a high probability. There are several ways of implementing such
$\epsilon\text{-\emph{decay}}$. We choose the rather simple option of choosing a random action when in state $s_t=s$ with probability 
$\epsilon_t(s_t) = \frac{c}{\#[(s=s_t)]}$ for some positive constant $c \leq 1$, though it should be noted that more advanced options 
such as Boltzmann exploration exist \citep{RLDP}.

\subsection{Software implementation} \label{sec:software}
Our initial idea for implementing a Q-learning algorithm for Blackjack was to find a pre-existing code base that
could simulate a Blackjack environment, enabling us to focus mostly on the reinforcement learning aspect of
the project. AI Gym (\url{https://gym.openai.com/}) is ``a toolkit for developing and comparing reinforcement learning 
algorithms'', which can be installed as a Python library and which offers a simple Blackjack environment---indeed,
seemingly the same environment as that in \citet{Sutton2018}. In this version of Blackjack, cards are dealt from
an infinite deck and one is provided the state representation 
$(\text{player sum},\text{dealer's 1 showing card}, \text{player has usable ace})$, much like that described
in Section \ref{sec:nonstationaryMDP}. On closer inspection of the source code, however, we found that this
environment was lacking.
With only the (first) visible card representing the dealer in any state, some state transitions will go 
from one state (call it $s$) to itself when the episode ends (and a reward is given). This means that some terminal
states are missing, and further that the action `hit' may be permissible at the first entry into $s$ but not
the subsequent one. For these reasons, $Q(s,a)$ will be given an erroneous value in its updates for these states. 
We also found the infinite deck setting to be less interesting, 
since it makes counting cards impossible.

Thus, a somewhat large effort was put into improving the Blackjack environment class from AI Gym. We replaced the
dealer's first visible card in the state representation by the dealer's visible card sum, made it possible to
have an integer number of decks in addition to infinitely many, and also made some other smaller changes.
These changes were made with inheritance in mind, as our next effort was spent on making a subclass of this
base class; the subclass implementing the extended state space discussed in \ref{sec:stationaryMDP}.
We have made our implementation publicly available at \url{https://github.com/Ethorsn/SF2957_project} and
are considering a submission (pull request) of it to AI Gym. 